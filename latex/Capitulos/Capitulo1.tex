\chapter{Referencial Teórico}

O trabalho feito na Universidade de Gotemburgo, na Suéćia, tem como bases duas 
áreas distintas, a biologia e a ciência da computação. Na área biológica, o foco 
é sobre os glicanos, biopolímeros com inúmeras aplicações e funções biológicas. 
Já na computação, foi estudado o uso de expressões regulares 
(\abrv[ReGex -- Expressões Regulares]{RegEx}), padrões formados por sequências 
de caracteres utilizados para busca, análise e manipulação de textos. Dessa 
forma, os pesquisadores aplicaram expressões regulares para formalizar uma 
estrutura de busca voltada à identificação de padrões específicos dos glicanos.

\section{Glicanos: Complexidade e Importância Biológica}

Para compreender melhor a abordagem biológica do estudo, é importante explorar 
os glicanos, suas características e funções, bem como outros conceitos 
relacionados abordados no trabalho.

\begin{definition}[Definição de Glicanos]
  Glicanos são polissacarídeos estruturais, longas cadeias formadas por unidades 
  de açúcar (monossacarídeos) ligadas entre si por ligações glicosídicas, presentes 
  abundantemente na Terra como componentes importantes de estruturas como 
  glicoproteínas, glicolipídeos e proteoglicanos, além de fazerem parte de 
  paredes celulares de fungos e leveduras.
\end{definition}

Essa diversidade de funções ocorre porque, em sua composição, existem 
subestruturas chamadas de motivos de glicanos, uma sequência ou arranjo 
particular de açúcares que funciona como um sinal de reconhecimento molecular, 
ou seja, a parte que carrega o significado mais importante e que é reconhecida 
por outras moléculas. São eles que trazem a importância biológica dos glicanos 
e, por isso, é de extrema importância que eles sejam entendidos pela ciência. 

Contudo, a grande diversidade e versatilidade traz à estrutura interesse 
contínuo de pesquisas de diversas áreas, como a imunologia, biotecnologia e 
parasitologia. Porém, como existem inúmeros motivos de glicanos, catalogá-los e 
reconhecê-los é um desafio.

\section{RegEx na Ciência da Computação}

Na ciência da computação, RegEx são uma sequência de caracteres que define um 
padrão de busca. Elas representam uma linguagem formal que pode ser utilizada 
para identificar, extrair e manipular subconjuntos de texto com base em regras e 
padrões específicos.

Essa notação envolve uma combinação de cadeias de símbolos do alfabeto da 
linguagem. Por exemplo, o operador main (+) é utilizado para denotar união, o 
asterisco (*) para o fecho estrela, indica zero ou mais ocorrências do caractere 
anterior, e o ponto (.) para concatenação. Essa sintaxe permite a criação de 
padrões que vão desde buscas simples, como encontrar uma palavra, até a validação 
de estruturas complexas em um texto.

\section{A Aplicação Inovadora de RegEx em Glicanos}

Diante do desafio  na análise dos glicanos, o estudo propôs a aplicação de 
expressões regulares como ferramenta para buscar e extrair seus padrões estruturais 
de forma precisa e flexível. Essa técnica foi adaptada no trabalho para lidar com a 
estrutura ramificada e não linear dos glicanos, permitindo uma análise muito mais 
eficiente e escalável. O artigo demonstra que, embora as aplicações tradicionais de 
RegEx sejam conhecidas, sua versatilidade permite o uso em contextos menos 
convencionais, como a bioinformática.
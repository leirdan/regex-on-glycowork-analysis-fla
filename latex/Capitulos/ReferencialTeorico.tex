% Introdução
\chapter{Referencial Teórico}

O trabalho feito na Universidade de Gotemburgo, na Suéćia, tem como bases duas áreas distintas, a biologia e a ciência da computação. Na área biológica, o foco é sobre os glicanos, biopolímeros com inúmeras aplicações e funções biológicas. Já na computação, foi estudado o uso de expressões regulares (ReGex), padrões formados por sequências de caracteres utilizados para busca, análise e manipulação de textos. Dessa forma, os pesquisadores aplicaram expressões regulares para formalizar uma estrutura de busca voltada à identificação de padrões específicos dos glicanos.

\section{Glicanos: Complexidade e Importância Biológica}

Para compreender melhor a abordagem biológica do estudo, é importante explorar os glicanos, suas características e funções, bem como outros conceitos relacionados abordados no trabalho. Glicanos são polissacarídeos estruturais, longas cadeias formadas por unidades de açúcar (monossacarídeos) ligadas entre si por ligações glicosídicas, presentes abundantemente na Terra como componentes importantes de estruturas como glicoproteínas, glicolipídeos e proteoglicanos, além de fazerem parte de paredes celulares de fungos e leveduras.

Essa diversidade de funções ocorre porque, em sua composição, existem subestruturas chamadas de motivos de glicanos, uma sequência ou arranjo particular de açúcares que funciona como um sinal de reconhecimento molecular, ou seja, a parte que carrega o significado mais importante e que é reconhecida por outras moléculas. São eles que trazem a importância biológica dos glicanos e, por isso, é de extrema importância que eles sejam entendidos pela ciência. 

Contudo, a grande diversidade e versatilidade traz à estrutura interesse contínuo de pesquisas de diversas áreas, como a imunologia, biotecnologia e parasitologia. Porém, como existem inúmeros motivos de glicanos, catalogá-los e reconhecê-los é um desafio.

\section{Expressões Regulares (RegEx) na Ciência da Computação}

\section{A Aplicação Inovadora de RegEx em Glicanos}
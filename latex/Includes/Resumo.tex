% Resumo em língua vernácula
\begin{center}
	{\Large{\textbf{Análise do Glycan RegEx e o uso de expressões regulares na biblioteca Glycowork}}}
\end{center}

\vspace{1cm}

\begin{flushright}
	Autores: Andriel Vinicius de Medeiros Fernandes, \\
	Gabrielle de Vasconcelos Borja, \\
	Jeremias Pinheiro de Araújo Andrade, \\
	Lucas Vinicius Dantas de Medeiros, \\
	María Paz Marcato, \\
	Ramon Cândido Jales de Barros \\
\end{flushright}

\vspace{1cm}

\begin{center}
	\Large{\textsc{\textbf{Resumo}}}
\end{center}

\noindent Este trabalho apresenta uma análise do artigo "Syntactic sugars: 
crafting a regular expression framework for glycan structures" de Bennett e 
Bojar (2024), que introduz o Glycan RegEx, um sistema para a aplicação de 
expressões regulares (RegEx) na busca por padrões em glicanos. O estudo investiga 
como essa abordagem, implementada na biblioteca glycowork, adapta os conceitos 
clássicos da ciência da computação para resolver o desafio de identificar subestruturas 
(motifs) na complexa e não linear topologia dos glicanos. A análise detalha o 
funcionamento do Glycan RegEx, que traduz a sintaxe de expressões regulares para 
operações de isomorfismo de subgrafos, permitindo uma busca declarativa e flexível. 
São examinadas funções-chave do processo, que particionam, padronizam e executam a 
busca dos padrões na estrutura molecular. Conclui-se que o trabalho de Bennett e 
Bojar demonstra com sucesso a versatilidade das expressões regulares para além do 
contexto textual, oferecendo uma ferramenta robusta para a análise em bioinformática 
e superando as limitações de métodos tradicionais baseados em bancos de motifs estáticos.

\noindent\textit{Palavras-chave}: Glycano, RegEx.